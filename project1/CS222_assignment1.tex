\documentclass[12pt,a4paper]{article}
\usepackage{ctex}
\usepackage{float}
\usepackage{amsmath,amscd,amsbsy,amssymb,latexsym,url,bm,amsthm}
\usepackage{epsfig,graphicx,subfigure}
\usepackage{enumitem,balance,mathtools}
\usepackage{wrapfig}
\usepackage{mathrsfs, euscript}
\usepackage[usenames]{xcolor}
\usepackage{hyperref}
%\usepackage{algorithm}
\usepackage[T1]{fontenc}
\usepackage{algorithmic}
%\usepackage[vlined,ruled,commentsnumbered,linesnumbered]{algorithm2e}
\usepackage[ruled,lined,boxed,linesnumbered]{algorithm2e}
\usepackage{xcolor}
\usepackage{listings}
%\floatname{algorithm}{Solution}
\newtheorem{theorem}{Theorem}[section]
\newtheorem{lemma}[theorem]{Lemma}
\newtheorem{proposition}[theorem]{Proposition}
\newtheorem{corollary}[theorem]{Corollary}
\newtheorem{exercise}{Exercise}[section]
\newtheorem*{solution}{Solution}

\renewcommand{\thefootnote}{\fnsymbol{footnote}}

\newcommand{\postscript}[2]
 {\setlength{\epsfxsize}{#2\hsize}
  \centerline{\epsfbox{#1}}}

\renewcommand{\baselinestretch}{1.0}

\setlength{\oddsidemargin}{-0.365in}
\setlength{\evensidemargin}{-0.365in}
\setlength{\topmargin}{-0.3in}
\setlength{\headheight}{0in}
\setlength{\headsep}{0in}
\setlength{\textheight}{10.1in}
\setlength{\textwidth}{7in}
\makeatletter \renewenvironment{proof}[1][Proof] {\par\pushQED{\qed}\normalfont\topsep6\p@\@plus6\p@\relax\trivlist\item[\hskip\labelsep\bfseries#1\@addpunct{.}]\ignorespaces}{\popQED\endtrivlist\@endpefalse} \makeatother
\makeatletter
\renewenvironment{solution}[1][Solution] {\par\pushQED{\qed}\normalfont\topsep6\p@\@plus6\p@\relax\trivlist\item[\hskip\labelsep\bfseries#1\@addpunct{.}]\ignorespaces}{\popQED\endtrivlist\@endpefalse} \makeatother
\begin{document}
\noindent

\lstset{numbers=left,numberstyle=\tiny,commentstyle=\color{red!50!green!50!blue!50},frame=shadowbox, rulesepcolor=\color{red!20!green!20!blue!20},escapeinside=``,xleftmargin=2em,xrightmargin=2em, aboveskip=1em,breaklines=true}
%\floatname{algorithm}{Solution}
%========================================================================
\noindent\framebox[\linewidth]{\shortstack[c]{
\Large{\textbf{OS Project 1}}\vspace{1mm}\\
Project for Computer Architecture \& Operating Systems by Chentao Wu, 2016 Autumn Semester}}

~\\
~\\
\begin{large}
\textbf{Project Step:}
\end{large}
~\\
~\\
\textbf{Part1 Preparation}
\begin{enumerate}

\item Build Ubuntu 12.04

\quad The version of Ubuntu 12.04 kernel is 3.2.0-23.

\item Download linux-2.6.38

\quad Download kernel from http://www.kernel.org. I used the 2.6.38 version because I have heard that it is easier to manipulate.

\item Open termination, input command in termination, get root power.
\begin{lstlisting}[language=bash]
sudo su
\end{lstlisting} 

\end{enumerate}
~\\
\textbf{Part2 Add a system call to the linux kernel}
\begin{enumerate}

\item Unzip the kernel
\begin{lstlisting}[language=bash]
mv linux-2.6.38.tar.bz2 /usr/src/
cd /usr/src
tar -jxvf linux-2.6.38.tar.bz2
\end{lstlisting} 

\quad Move the zip file to the "/usr/src"directory, where needs the root power. Then unzip it.

\item Add the system call
\begin{lstlisting}[language=bash]
cd linux-2.6.38/kernel
gedit sys.c
\end{lstlisting}
\quad open sys.c file, add header at the beginning of the file:
\begin{lstlisting}[language=C]
#include<linux/linkage.h>
#include<linux/kernel.h>
\end{lstlisting}
\quad Add the code at the end of the file
\begin{lstlisting}[language=C]
asmlinkage int sys_helloworld(){
	printk(KERN_EMERG "hello world!(by Yaowei Huang,Nov.2016)");
	return 1;
}
\end{lstlisting}
\quad And there I have made a small mistake..It's October now...

\item Modify the pointer list
\begin{lstlisting}[language=bash]
cd /usr/src/linux-2.6.38/arch/x86/kernel
gedit syscall_table_32.S
\end{lstlisting} 
\quad open syscall\_table\_32.S file, add the code at the end of it.
\begin{lstlisting}[language=C]
.long sys_helloworld
\end{lstlisting}
\begin{lstlisting}[language=bash]
cd /usr/src/linux-2.6.38/arch/x86/include/asm
gedit unistd_32.h
\end{lstlisting}
\quad open unistd\_32.h file, at the end of it add the code:
\begin{lstlisting}[language=C]
#define __NR_helloworld     341
\end{lstlisting}
\quad And change the code:
\begin{lstlisting}[language=C]
#define __NR_syscalls     341
\end{lstlisting}
\quad to
\begin{lstlisting}[language=C]
#define __NR_syscalls     342
\end{lstlisting}
\end{enumerate}
~\\
\textbf{Part 3 Compile the kernel}
\begin{enumerate}
\item Preparaion
\begin{lstlisting}[language=bash]
cd /usr/src/linux-2.6.38
apt-get install build-essential kernel-package libncurses5-dev fakeroot
\end{lstlisting}
\item Compile
\begin{lstlisting}[language=bash]
make mrproper
//it is alternative, to clean the compiler history
make menuconfig
//to generate a .config file
make -j4
//the command of compiling
\end{lstlisting}
\quad And then it was a long time to wait, I waited about two and a half hours.
\end{enumerate}
~\\
\textbf{Part 4 Install the kernel}
\begin{enumerate}
\item Install kernel
\begin{lstlisting}[language=bash]
make modules_install
make install
\end{lstlisting}
\item Make the new kernel to be loaded
\begin{lstlisting}[language=bash]
sudo mkinitramfs -o /book/initrd.img-2.6.38
sudo update-initramfs -c -k 2.6.38
sudo update-grub2
\end{lstlisting}
\quad The first command will create files like initrd.img-2.6.38 etc. under /boot list.

\quad The second command will update the files corresponding to /lib/modules/2.6.32.63 files.
\item Add chice to the grup list

\quad Change the "GRUB\_HIDDEN\_TIMEOUT=0" in the file of /etc/default/grub to "GRUB\_HIDDEN\_TIMEOUT=4"

\quad Change all "set timeout=0" in the file of /etc/grub.d/30\_os-prober to "set timeout=10".

\quad Update grup2
\begin{lstlisting}[language=bash]
sudo update-grub2
\end{lstlisting}
\end{enumerate}
~\\
\textbf{Part 5 Reboot}

\quad Reboot the ubuntu and choose to enter the 2.6.38 version kernel.

\begin{large}
~\\
~\\
\textbf{Validate}
\end{large}

\begin{lstlisting}[language=C] 
#include<stdio.h>
#include<unistd.h>
#include<sys/syscall.h>

#define SYS_helloworld   341   
//same as defined before

 int main(){
	int tmp;
	tmp=syscall(341);
	printf("\n");
	if(tmp==1){
		printf("success!\n");
	}
	return 0;
}
\end{lstlisting}

\quad Input command in termination 

\begin{lstlisting}[language=bash] 
gcc a.c
ll a.c a.out
./a.out
dmesg -d
\end{lstlisting}
\begin{large}
~\\
~\\
\textbf{Result}
\end{large}

The pictrue of result is as follows:

\begin{figure}[H]
  \centering
  \includegraphics[width=1\textwidth]{01.png}
  \caption{Result}%\label{fig:digit}
\end{figure}

\begin{large}
~\\
~\\
\textbf{The problem I have met}
\end{large}

The most difficult problem is the following error: 

\quad\quad .size expression for apf\_fault does not evaluate to a constant

\begin{figure}[H]
  \centering
  \includegraphics[width=1\textwidth]{02.png}
  \caption{Problem}%\label{fig:digit}
\end{figure}

And at first I didn't find this problem, because I was not using the command "make bzImage", then I found that the command "make -j4" runs very fast, which only needed about 10 minutes, much shorter than my roommates. Later I found I can't "make install".

\begin{figure}[H]
  \centering
  \includegraphics[width=1\textwidth]{03.png}
  \caption{Problem}%\label{fig:digit}
\end{figure}

So I think there must be something wrong, and I searched the website to find more command to try, until I tried the "make bzImage" command to find the problem.

  Fortunately, there's someone having solved this problem. I entered the website

\quad\quad\quad\quad\quad http://blog.csdn.net/baozhb/article/details/9005150

The solution is to modify two lines of codes (code mistake) in /.../arch/x86/kernel/entry\_32.S. And then the problem is solved quickly.

What's more, I also have met a stupid problem that I didn't allocate enough virtual disks space to my Ubuntu...

\begin{large}
~\\
~\\
\textbf{Harvest}
\end{large}

Through this project I learned a lot knowledge about the kernel and how to add a  system call. Although I haven't done as the same as what the book recommend, but I think my way is as good too, which is more simple. The operating system is a very fun thing to manipulate, and solving problems makes me very proudable.
%========================================================================
\end{document}
